Plotting the BFTLE fields to see if they give us more insight into the underlying velocity field governing the motion of ants is the next task. Following that, I want to experiment with different initial conditions (like pre-existing circular trails, initial conditions along pre-existing trail, etc.) to see if ant-mill formation can be replicated using the current model for these special initial conditions. 

Following these, I want to tweak the model a bit, reduce randomness, and see whether milling is a asymptotic solution to any model governing ant interaction with pheromone deposition and update defined as we already have. This would still not fully describe ant motion in isolation, since observations of ant milling claim milling to be a transient phenomena, with the mill dispersing after some time (unless the ants die of exhaustion during the mill). Observing transient vortices in the FTLE fields for certain models would also provide some insight into what the underlying velocity field for ant motion looks like.